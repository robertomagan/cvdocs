%\documentclass[a4paper,11pt]{book}
\documentclass[a4paper,twoside,11pt,titlepage]{book}
\usepackage{listings}
\usepackage[utf8]{inputenc}
\usepackage[spanish]{babel}


% \usepackage[style=list, number=none]{glossary} %
%\usepackage{titlesec}
%\usepackage{pailatino}

\decimalpoint
\usepackage{dcolumn}
\newcolumntype{.}{D{.}{\esperiod}{-1}}
\makeatletter
\addto\shorthandsspanish{\let\esperiod\es@period@code}
\makeatother


%\usepackage[chapter]{algorithm}
\RequirePackage{verbatim}
%\RequirePackage[Glenn]{fncychap}
\usepackage{fancyhdr}
\usepackage{graphicx}
\usepackage{afterpage}
\usepackage{enumerate}
\usepackage{cite}
\usepackage{longtable}
\usepackage{float}

\setcounter{tocdepth}{4}
\setcounter{secnumdepth}{4}

\usepackage[pdfborder={000}]{hyperref} %referencia

% ********************************************************************
% Re-usable information
% ********************************************************************
\newcommand{\myTitle}{Título del proyecto\xspace}
\newcommand{\myDegree}{Grado en ...\xspace}
\newcommand{\myName}{Nombre Apllido1 Apellido2 (alumno)\xspace}
\newcommand{\myProf}{Nombre Apllido1 Apellido2 (tutor1)\xspace}
\newcommand{\myOtherProf}{Nombre Apllido1 Apellido2 (tutor2)\xspace}
%\newcommand{\mySupervisor}{Put name here\xspace}
\newcommand{\myFaculty}{Escuela Técnica Superior de Ingenierías Informática y de
Telecomunicación\xspace}
\newcommand{\myFacultyShort}{E.T.S. de Ingenierías Informática y de
Telecomunicación\xspace}
\newcommand{\myDepartment}{Departamento de ...\xspace}
\newcommand{\myUni}{\protect{Universidad de Granada}\xspace}
\newcommand{\myLocation}{Granada\xspace}
\newcommand{\myTime}{\today\xspace}
\newcommand{\myVersion}{Version 0.1\xspace}


\hypersetup{
pdfauthor = {\myName (email (en) ugr (punto) es)},
pdftitle = {\myTitle},
pdfsubject = {},
pdfkeywords = {palabra_clave1, palabra_clave2, palabra_clave3, ...},
pdfcreator = {LaTeX con el paquete ....},
pdfproducer = {pdflatex}
}

%\hyphenation{}


%\usepackage{doxygen/doxygen}
%\usepackage{pdfpages}
\usepackage{url}
\usepackage{colortbl,longtable}
\usepackage[stable]{footmisc}

%\usepackage{index}

%\makeindex
%\usepackage[style=long, cols=2,border=plain,toc=true,number=none]{glossary}
% \makeglossary

% Definición de comandos que me son tiles:
%\renewcommand{\indexname}{Índice alfabético}
%\renewcommand{\glossaryname}{Glosario}

\pagestyle{fancy}
\fancyhf{}
\fancyhead[LO]{\leftmark}
\fancyhead[RE]{\rightmark}
\fancyhead[RO,LE]{\textbf{\thepage}}
\renewcommand{\chaptermark}[1]{\markboth{\textbf{#1}}{}}
\renewcommand{\sectionmark}[1]{\markright{\textbf{\thesection. #1}}}

\setlength{\headheight}{1.5\headheight}

\newcommand{\HRule}{\rule{\linewidth}{0.5mm}}
%Definimos los tipos teorema, ejemplo y definición podremos usar estos tipos
%simplemente poniendo \begin{teorema} \end{teorema} ...
\newtheorem{teorema}{Teorema}[chapter]
\newtheorem{ejemplo}{Ejemplo}[chapter]
\newtheorem{definicion}{Definición}[chapter]

\definecolor{gray97}{gray}{.97}
\definecolor{gray75}{gray}{.75}
\definecolor{gray45}{gray}{.45}
\definecolor{gray30}{gray}{.94}

\lstset{ frame=Ltb,
     framerule=0.5pt,
     aboveskip=0.5cm,
     framextopmargin=3pt,
     framexbottommargin=3pt,
     framexleftmargin=0.1cm,
     framesep=0pt,
     rulesep=.4pt,
     backgroundcolor=\color{gray97},
     rulesepcolor=\color{black},
     %
     stringstyle=\ttfamily,
     showstringspaces = false,
     basicstyle=\scriptsize\ttfamily,
     commentstyle=\color{gray45},
     keywordstyle=\bfseries,
     %
     numbers=left,
     numbersep=6pt,
     numberstyle=\tiny,
     numberfirstline = false,
     breaklines=true,
   }
 
% minimizar fragmentado de listados
\lstnewenvironment{listing}[1][]
   {\lstset{#1}\pagebreak[0]}{\pagebreak[0]}

\lstdefinestyle{CodigoC}
   {
	basicstyle=\scriptsize,
	frame=single,
	language=C,
	numbers=left
   }
\lstdefinestyle{CodigoC++}
   {
	basicstyle=\small,
	frame=single,
	backgroundcolor=\color{gray30},
	language=C++,
	numbers=left
   }

 
\lstdefinestyle{Consola}
   {basicstyle=\scriptsize\bf\ttfamily,
    backgroundcolor=\color{gray30},
    frame=single,
    numbers=none
   }


\newcommand{\bigrule}{\titlerule[0.5mm]}


%Para conseguir que en las páginas en blanco no ponga cabecerass
\makeatletter
\def\clearpage{
  \ifvmode
    \ifnum \@dbltopnum =\m@ne
      \ifdim \pagetotal <\topskip
        \hbox{}
      \fi
    \fi
  \fi
  \newpage
  \thispagestyle{empty}
  \write\m@ne{}
  \vbox{}
  \penalty -\@Mi
}
\makeatother

\usepackage{pdfpages}
% pdfpages package global options
% pages=- , means that whole pdf document is included
% nup=2x2 , means that 4 pdf are include in one document page
% linktodoc=true, insert a link to the document (name of the document must finish with *.pdf)
\includepdfset{pages=-, pagecommand=\thispagestyle{plain}, openright=false, linktodoc=true}
\begin{document}
\begin{titlepage}
 
 
\newlength{\centeroffset}
\setlength{\centeroffset}{-0.5\oddsidemargin}
\addtolength{\centeroffset}{0.5\evensidemargin}
\thispagestyle{empty}

\noindent\hspace*{\centeroffset}\begin{minipage}{\textwidth}

\centering
\includegraphics[width=0.9\textwidth]{imagenes/logo_ugr.jpg}\\[1.4cm]

\textsc{ \Large MÉRITOS APORTADOS POR EL CANDIDATO\\[0.2cm]}
\textsc{(PAD y PCD)}\\[0.5cm]
% Upper part of the page
% 
% Title
{\Huge\bfseries Nombre Apellido1 Apellido2\\
}
\noindent\rule[-1ex]{\textwidth}{2pt}\\[1ex]

\end{minipage}

\vspace{2cm}
\noindent\hspace*{\centeroffset}\begin{minipage}{\textwidth}
\centering

\includegraphics[width=0.3\textwidth]{imagenes/citic.png}\\[0.1cm]
\textsc{Centro de trabajo}\\
\textsc{---}\\
Dirección\\
18071 Granada\\
\textsc{---}\\
%\includegraphics[width=0.2\textwidth]{imagenes/tstc.png}\\[0.1cm]
%\textsc{Dpto. de Teoría de la Señal, Telemática y Comunicaciones}\\
%\textsc{---}\\
\end{minipage}
%\addtolength{\textwidth}{\centeroffset}
%\vspace{\stretch{2}}
\end{titlepage}



%
%\thispagestyle{empty}
%\cleardoublepage

%\thispagestyle{empty}

\begin{titlepage}
 
 
\setlength{\centeroffset}{-0.5\oddsidemargin}
\addtolength{\centeroffset}{0.5\evensidemargin}
\thispagestyle{empty}

\noindent\hspace*{\centeroffset}\begin{minipage}{\textwidth}

\centering
%\includegraphics[width=0.9\textwidth]{imagenes/logo_ugr.jpg}\\[1.4cm]

%\textsc{ \Large PROYECTO FIN DE CARRERA\\[0.2cm]}
%\textsc{ INGENIERÍA EN INFORMÁTICA}\\[1cm]
% Upper part of the page
% 

 \vspace{3.3cm}

%si el proyecto tiene logo poner aquí



% Title

{\Huge\bfseries  Calibración de Modelos Multivariantes en Big Data con Python y Spark\\
}
\noindent\rule[-1ex]{\textwidth}{3pt}\\[3.5ex]

\end{minipage}

\vspace{2cm}
\noindent\hspace*{\centeroffset}\begin{minipage}{\textwidth}
\centering

\textbf{Autor}\\ {Francisco José Álvarez Márquez}\\[2.5ex]
\textbf{Directores}\\
{Nombre Apellido1 Apellido2 (tutor1)\\
Nombre Apellido1 Apellido2 (tutor2)}\\[2cm]
\includegraphics[width=0.15\textwidth]{imagenes/tstc.png}\\[0.1cm]
\textsc{Departamento de Teoría de la Señal, Telemática y Comunicaciones}\\
\textsc{---}\\
Granada, mes de 201
\end{minipage}
%\addtolength{\textwidth}{\centeroffset}
\vspace{\stretch{2}}

 
\end{titlepage}






\cleardoublepage
\thispagestyle{empty}

\begin{center}
{\large\bfseries  Calibración de Modelos Multivariantes en Big Data con Python y Spark}\\
\end{center}
\begin{center}
Francisco José Álvarez Márquez\\
\end{center}

%\vspace{0.7cm}
\noindent{\textbf{Palabras clave}: palabra\_clave1, palabra\_clave2, palabra\_clave3, ......}\\

\vspace{0.7cm}
\noindent{\textbf{Resumen}}\\

Poner aquí el resumen.
\cleardoublepage


\thispagestyle{empty}


\begin{center}
{\large\bfseries Calibración de Modelos Multivariantes en Big Data con Python y Spark}\\
\end{center}
\begin{center}
Francisco José Álvarez Márquez\\
\end{center}

%\vspace{0.7cm}
\noindent{\textbf{Keywords}: Keyword1, Keyword2, Keyword3, ....}\\

\vspace{0.7cm}
\noindent{\textbf{Abstract}}\\

Write here the abstract in English.


\thispagestyle{empty}

\noindent\rule[-1ex]{\textwidth}{2pt}\\[4.5ex]

Yo, \textbf{Nombre Apellido1 Apellido2}, alumno de la titulación TITULACIÓN de la \textbf{Escuela Técnica Superior
de Ingenierías Informática y de Telecomunicación de la Universidad de Granada}, con DNI XXXXXXXXX, autorizo la
ubicación de la siguiente copia de mi Trabajo Fin de Grado en la biblioteca del centro para que pueda ser
consultada por las personas que lo deseen.

\vspace{6cm}

\noindent Fdo: Nombre Apellido1 Apellido2

\vspace{2cm}

\begin{flushright}
Granada a X de mes de 201 .
\end{flushright}



\thispagestyle{empty}

\noindent\rule[-1ex]{\textwidth}{2pt}\\[4.5ex]

D. \textbf{Nombre Apellido1 Apellido2 (tutor1)}, Profesor del Área de XXXX del Departamento YYYY de la Universidad de Granada.

\vspace{0.5cm}

D. \textbf{Nombre Apellido1 Apellido2 (tutor2)}, Profesor del Área de XXXX del Departamento YYYY de la Universidad de Granada.


\vspace{0.5cm}

\textbf{Informan:}

\vspace{0.5cm}

Que el presente trabajo, titulado \textit{\textbf{Título del proyecto, Subtítulo del proyecto}},
ha sido realizado bajo su supervisión por \textbf{Nombre Apellido1 Apellido2 (alumno)}, y autorizamos la defensa de dicho trabajo ante el tribunal
que corresponda.

\vspace{0.5cm}

Y para que conste, expiden y firman el presente informe en Granada a X de mes de 201 .

\vspace{1cm}

\textbf{Los directores:}

\vspace{5cm}

\noindent \textbf{Nombre Apellido1 Apellido2 (tutor1) \ \ \ \ \ Nombre Apellido1 Apellido2 (tutor2)}

\chapter*{Agradecimientos}
\thispagestyle{empty}

       \vspace{1cm}


Poner aquí agradecimientos...


\frontmatter
\tableofcontents
%\listoffigures
%\listoftables

\mainmatter
\setlength{\parskip}{-2pt}

\chapter{EXPERIENCIA INVESTIGADORA}
\section{Publicaciones científicas}
\subsection{Artículos en revistas}
\includepdf{/home/roberto/RMAGAN/nesgownCloud/INVESTIGACION/PROYECTOS/Nacionales/VERITAS/DESARROLLO/SENSOR/workspaceclipse/cvdocs/src/examples/documents/blank.pdf}
\includepdf{/home/roberto/RMAGAN/nesgownCloud/INVESTIGACION/PROYECTOS/Nacionales/VERITAS/DESARROLLO/SENSOR/workspaceclipse/cvdocs/src/examples/documents/experiencia_investigadora/0_publicaciones_cientificas/0_articulos_en_revistas/[1]_example.pdf}
\includepdf{/home/roberto/RMAGAN/nesgownCloud/INVESTIGACION/PROYECTOS/Nacionales/VERITAS/DESARROLLO/SENSOR/workspaceclipse/cvdocs/src/examples/documents/experiencia_investigadora/0_publicaciones_cientificas/0_articulos_en_revistas/[2]_example.pdf}
\subsection{Libros y capítulos de libros}
\includepdf{/home/roberto/RMAGAN/nesgownCloud/INVESTIGACION/PROYECTOS/Nacionales/VERITAS/DESARROLLO/SENSOR/workspaceclipse/cvdocs/src/examples/documents/blank.pdf}
\includepdf{/home/roberto/RMAGAN/nesgownCloud/INVESTIGACION/PROYECTOS/Nacionales/VERITAS/DESARROLLO/SENSOR/workspaceclipse/cvdocs/src/examples/documents/experiencia_investigadora/0_publicaciones_cientificas/1_libros_y_capitulos_de_libros/[3]_example.pdf}
\includepdf{/home/roberto/RMAGAN/nesgownCloud/INVESTIGACION/PROYECTOS/Nacionales/VERITAS/DESARROLLO/SENSOR/workspaceclipse/cvdocs/src/examples/documents/experiencia_investigadora/0_publicaciones_cientificas/1_libros_y_capitulos_de_libros/[4]_example.pdf}
\section{Participación en proyectos de investigación y/o en contratos de I+D}
\includepdf{/home/roberto/RMAGAN/nesgownCloud/INVESTIGACION/PROYECTOS/Nacionales/VERITAS/DESARROLLO/SENSOR/workspaceclipse/cvdocs/src/examples/documents/blank.pdf}
\includepdf{/home/roberto/RMAGAN/nesgownCloud/INVESTIGACION/PROYECTOS/Nacionales/VERITAS/DESARROLLO/SENSOR/workspaceclipse/cvdocs/src/examples/documents/experiencia_investigadora/1_participacion_en_proyectos_contratos_i_mas_d/[2]_example.pdf}
\includepdf{/home/roberto/RMAGAN/nesgownCloud/INVESTIGACION/PROYECTOS/Nacionales/VERITAS/DESARROLLO/SENSOR/workspaceclipse/cvdocs/src/examples/documents/experiencia_investigadora/1_participacion_en_proyectos_contratos_i_mas_d/[4]_example.pdf}
\includepdf{/home/roberto/RMAGAN/nesgownCloud/INVESTIGACION/PROYECTOS/Nacionales/VERITAS/DESARROLLO/SENSOR/workspaceclipse/cvdocs/src/examples/documents/blank.pdf}
\section{Contribuciones a congresos y conferencias científicas}
\includepdf{/home/roberto/RMAGAN/nesgownCloud/INVESTIGACION/PROYECTOS/Nacionales/VERITAS/DESARROLLO/SENSOR/workspaceclipse/cvdocs/src/examples/documents/blank.pdf}
\includepdf{/home/roberto/RMAGAN/nesgownCloud/INVESTIGACION/PROYECTOS/Nacionales/VERITAS/DESARROLLO/SENSOR/workspaceclipse/cvdocs/src/examples/documents/experiencia_investigadora/2_congresos_y_conferencias/[1]_example.pdf}
\includepdf{/home/roberto/RMAGAN/nesgownCloud/INVESTIGACION/PROYECTOS/Nacionales/VERITAS/DESARROLLO/SENSOR/workspaceclipse/cvdocs/src/examples/documents/experiencia_investigadora/2_congresos_y_conferencias/[2]_example.pdf}
\includepdf{/home/roberto/RMAGAN/nesgownCloud/INVESTIGACION/PROYECTOS/Nacionales/VERITAS/DESARROLLO/SENSOR/workspaceclipse/cvdocs/src/examples/documents/experiencia_investigadora/2_congresos_y_conferencias/[3]_example.pdf}
\includepdf{/home/roberto/RMAGAN/nesgownCloud/INVESTIGACION/PROYECTOS/Nacionales/VERITAS/DESARROLLO/SENSOR/workspaceclipse/cvdocs/src/examples/documents/experiencia_investigadora/2_congresos_y_conferencias/[4]_example.pdf}
\section{Otros méritos relevantes de investigación}
\includepdf{/home/roberto/RMAGAN/nesgownCloud/INVESTIGACION/PROYECTOS/Nacionales/VERITAS/DESARROLLO/SENSOR/workspaceclipse/cvdocs/src/examples/documents/blank.pdf}
\includepdf{/home/roberto/RMAGAN/nesgownCloud/INVESTIGACION/PROYECTOS/Nacionales/VERITAS/DESARROLLO/SENSOR/workspaceclipse/cvdocs/src/examples/documents/experiencia_investigadora/3_otros_meritos_investigacion/[1]_example.pdf}
\includepdf{/home/roberto/RMAGAN/nesgownCloud/INVESTIGACION/PROYECTOS/Nacionales/VERITAS/DESARROLLO/SENSOR/workspaceclipse/cvdocs/src/examples/documents/experiencia_investigadora/3_otros_meritos_investigacion/[2]_example.pdf}
\includepdf{/home/roberto/RMAGAN/nesgownCloud/INVESTIGACION/PROYECTOS/Nacionales/VERITAS/DESARROLLO/SENSOR/workspaceclipse/cvdocs/src/examples/documents/experiencia_investigadora/3_otros_meritos_investigacion/[4]_example.pdf}
\includepdf{/home/roberto/RMAGAN/nesgownCloud/INVESTIGACION/PROYECTOS/Nacionales/VERITAS/DESARROLLO/SENSOR/workspaceclipse/cvdocs/src/examples/documents/experiencia_investigadora/3_otros_meritos_investigacion/[5]_example.pdf}


\end{document}
